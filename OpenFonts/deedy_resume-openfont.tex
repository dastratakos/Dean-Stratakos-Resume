%%%%%%%%%%%%%%%%%%%%%%%%%%%%%%%%%%%%%%%
% Deedy - One Page Two Column Resume
% LaTeX Template
% Version 1.2 (16/9/2014)
%
% Original author:
% Debarghya Das (http://debarghyadas.com)
%
% Original repository:
% https://github.com/deedydas/Deedy-Resume
%
% IMPORTANT: THIS TEMPLATE NEEDS TO BE COMPILED WITH XeLaTeX
%
% This template uses several fonts not included with Windows/Linux by
% default. If you get compilation errors saying a font is missing, find the line
% on which the font is used and either change it to a font included with your
% operating system or comment the line out to use the default font.
% 
%%%%%%%%%%%%%%%%%%%%%%%%%%%%%%%%%%%%%%
% 
% TODO:
% 1. Integrate biber/bibtex for article citation under publications.
% 2. Figure out a smoother way for the document to flow onto the next page.
% 3. Add styling information for a "Projects/Hacks" section.
% 4. Add location/address information
% 5. Merge OpenFont and MacFonts as a single sty with options.
% 
%%%%%%%%%%%%%%%%%%%%%%%%%%%%%%%%%%%%%%
%
% CHANGELOG:
% v1.1:
% 1. Fixed several compilation bugs with \renewcommand
% 2. Got Open-source fonts (Windows/Linux support)
% 3. Added Last Updated
% 4. Move Title styling into .sty
% 5. Commented .sty file.
%
%%%%%%%%%%%%%%%%%%%%%%%%%%%%%%%%%%%%%%%
%
% Known Issues:
% 1. Overflows onto second page if any column's contents are more than the
% vertical limit
% 2. Hacky space on the first bullet point on the second column.
%
%%%%%%%%%%%%%%%%%%%%%%%%%%%%%%%%%%%%%%


\documentclass[]{deedy-resume-openfont}
\usepackage{fancyhdr}
\usepackage{fontawesome}
 
\pagestyle{fancy}
\fancyhf{}
 
\begin{document}

%%%%%%%%%%%%%%%%%%%%%%%%%%%%%%%%%%%%%%
%
%     LAST UPDATED DATE
%
%%%%%%%%%%%%%%%%%%%%%%%%%%%%%%%%%%%%%%
% \lastupdated

%%%%%%%%%%%%%%%%%%%%%%%%%%%%%%%%%%%%%%
%
%     TITLE NAME
%
%%%%%%%%%%%%%%%%%%%%%%%%%%%%%%%%%%%%%%
\namesection{}{Dean Stratakos}{ \urlstyle{same}
% Computer Science Student-Athlete at Stanford University\\
\faMapPin $ $ Saratoga, CA |
\faEnvelope $ $ \href{mailto:dstratak@stanford.edu}{dstratak@stanford.edu} | 
\faMobile $ $ (408) 797-4107
}

%%%%%%%%%%%%%%%%%%%%%%%%%%%%%%%%%%%%%%
%
%     COLUMN ONE
%
%%%%%%%%%%%%%%%%%%%%%%%%%%%%%%%%%%%%%%

\begin{minipage}[t]{0.33\textwidth} 

%%%%%%%%%%%%%%%%%%%%%%%%%%%%%%%%%%%%%%
%     EDUCATION
%%%%%%%%%%%%%%%%%%%%%%%%%%%%%%%%%%%%%%

\section{Education} 

\subsection{Stanford University}
\descript{BS in Computer Science}
\location{Sep 2018 - (exp) Jun 2022}
\location{Stanford, CA}
\location{GPA: 4.09 / 4.00}
\sectionsep

\subsection{Saratoga High School}
\location{Aug 2014 - Jun 2018}
\location{Saratoga, CA}
\location{GPA: 4.71 / 4.00}
\sectionsep

%%%%%%%%%%%%%%%%%%%%%%%%%%%%%%%%%%%%%%
%     LINKS
%%%%%%%%%%%%%%%%%%%%%%%%%%%%%%%%%%%%%%

\section{Links}
% \href{http://dastratakos.github.io}{
% \begin{tabular}{l l}
%     \faUser $ $ $ $ Website: & \bf dastratakos.github.io
% \end{tabular}
% }
% \href{https://github.com/dastratakos}{
% \begin{tabular}{l l}
%     \faGithub $ $ $ $ Github: & \bf dastratakos
% \end{tabular}
% }
% \href{https://www.linkedin.com/in/dean-stratakos-8b338b149/}{
% \begin{tabular}{l l}
%     \faLinkedin $ $ $ $ LinkedIn: & \bf dean-stratakos
% \end{tabular}
% }
\begin{tabular}{l l}
    \href{http://dastratakos.github.io}{\faUser $ $ $ $ Website:} &
    \href{http://dastratakos.github.io}{\bf dastratakos.github.io} \\
    
    \href{https://github.com/dastratakos}{\faGithub $ $ $ $ Github:} &
    \href{https://github.com/dastratakos}{\bf dastratakos} \\
    
    \href{https://www.linkedin.com/in/dean-stratakos-8b338b149/}{\faLinkedin $ $ $ $ LinkedIn:} &
    \href{https://www.linkedin.com/in/dean-stratakos-8b338b149/}{\bf dean-stratakos}
\end{tabular}
\sectionsep

%%%%%%%%%%%%%%%%%%%%%%%%%%%%%%%%%%%%%%
%     COURSEWORK
%%%%%%%%%%%%%%%%%%%%%%%%%%%%%%%%%%%%%%

\section{Coursework}
Algorithms and Data Structures \\
Artificial Intelligence \\
Computer Organization \& Systems \\
Machine Learning \\
\vspace{\topsep} % Hacky fix for awkward extra vertical space
\begin{tightemize}
  \item Convolutional Neural Networks
  \item Deep Learning
  \item Natural Language Processing
\end{tightemize}
Mathematics \\
\begin{tightemize}
  \item Linear Algebra
  \item Multivariable Calculus
  \item Probability
\end{tightemize}
Web Applications
\sectionsep

%%%%%%%%%%%%%%%%%%%%%%%%%%%%%%%%%%%%%%
%     SKILLS
%%%%%%%%%%%%%%%%%%%%%%%%%%%%%%%%%%%%%%

\section{Skills}
\subsection{Programming Languages}
\location{Proficient:}
Python \textbullet{} C++ \textbullet{} C \textbullet{} Java \textbullet{} JavaScript \\
HTML \textbullet{} CSS \textbullet{} Assembly \textbullet{} \LaTeX \\
\location{Familiar:}
Swift \textbullet{} Kotlin
\sectionsep

\subsection{Tools}
\location{Proficient:}
NumPy \textbullet{} scikit-learn \textbullet{} PyTorch \textbullet{} Unix \\
TensorFlow \textbullet{} Git/GitHub \textbullet{} SQLite \\
MongoDB (NoSQL) \textbullet{} Android Studio \\
\location{Familiar:}
AWS \textbullet{} Microsoft Azure \textbullet{} Pandas
\sectionsep

%%%%%%%%%%%%%%%%%%%%%%%%%%%%%%%%%%%%%%
%     Clubs and Interests
%%%%%%%%%%%%%%%%%%%%%%%%%%%%%%%%%%%%%%

\section{Clubs and Interests}
Stanford Christian Students \textbullet{} Surfing \\
Golf \textbullet{} Volunteering \textbullet{} Saturday Night Live

%%%%%%%%%%%%%%%%%%%%%%%%%%%%%%%%%%%%%%
%     Awards
%%%%%%%%%%%%%%%%%%%%%%%%%%%%%%%%%%%%%%

% \section{Awards}
% \descript{ITA Scholar Athlete \location{| 2019, 2020}}
% \descript{Pac-12 All-Academic Honor Roll \location{| 2019, 2020}}

%%%%%%%%%%%%%%%%%%%%%%%%%%%%%%%%%%%%%%
%
%     COLUMN TWO
%
%%%%%%%%%%%%%%%%%%%%%%%%%%%%%%%%%%%%%%

\end{minipage} 
\hfill
\begin{minipage}[t]{0.66\textwidth} 

%%%%%%%%%%%%%%%%%%%%%%%%%%%%%%%%%%%%%%
%     OBJECTIVE
%%%%%%%%%%%%%%%%%%%%%%%%%%%%%%%%%%%%%%

% \section{Objective}
% To obtain a summer internship for 2021 in which I can utilize and develop my knowledge in computer science. I am most interested in computer vision, machine learning, and big data.
% \sectionsep

%%%%%%%%%%%%%%%%%%%%%%%%%%%%%%%%%%%%%%
%     EXPERIENCE
%%%%%%%%%%%%%%%%%%%%%%%%%%%%%%%%%%%%%%

\section{Work Experience}
\runsubsection{Apple}
\descriptOneLine{| Software Engineering Intern,}
\descriptSmall{Advanced Computation Group}
\location{Oct 2020 - Present | Portland, OR (remote)}
\sectionsep

\runsubsection{Apple}
\descript{| Software Engineering Intern, Panic Triage team }
\location{Jun - Sep 2020 | Cupertino, CA (remote)}
\vspace{\topsep} % Hacky fix for awkward extra vertical space
\begin{tightemize}
    \item Improved the performance, scalability, and maintainability of a machine learning clustering algorithm that grouped duplicate kernel panic reports.
    \item Achieved cluster efficiency \href{https://en.wikipedia.org/wiki/Rand_index#:~:text=(true\%20negatives).-,Adjusted\%20Rand\%20index,specified\%20by\%20a\%20random\%20model.}{$\link{ARIs}$} of 84-89\% for 2 new panic signatures on iOS, macOS Apple Silicon, and macOS Intel.
    \item Concepts included agglomerative clustering, \href{https://en.wikipedia.org/wiki/Tf\%E2\%80\%93idf#:~:text=In\%20information\%20retrieval\%2C\%20tf\%E2\%80\%93idf,in\%20a\%20collection\%20or\%20corpus.}{$\link{tf-idf}$}, and cloud storage.
\end{tightemize}
\sectionsep

\runsubsection{\href{https://quadric.io/}{Quadric$\link{}$}}
\descript{(startup) | Software Engineering Intern }
\location{June – Aug 2019 | Burlingame, CA}
\begin{tightemize}
    \item Implemented the back end for 6 CNN layers in a C++ based intermediate language on a specialized edge-computing hardware architecture.
    \item Analyzed compile-time and run-time optimizations for a deep learning network.
    \item Studied post-training weight quantization to improve inference efficiency.
\end{tightemize}
\sectionsep

%%%%%%%%%%%%%%%%%%%%%%%%%%%%%%%%%%%%%%
%     Technical Projects
%%%%%%%%%%%%%%%%%%%%%%%%%%%%%%%%%%%%%%

\section{Technical Projects}
\runsubsection{Photo Sharing Web Application}
\descript{| CS 142}
\location{May - Jun 2020 |
Languages: JavaScript, HTML, CSS |
\href{https://youtu.be/mFzJm07WL5A}{\faYoutubePlay $ $ YouTube demo$\link{}$}}
\begin{tightemize}
    \item Developed a full stack ReactJS web application with a Node.js web server.
    \item Utilized a MongoDB database and Material-UI front end components.
\end{tightemize}
\sectionsep

% \runsubsection{Wikipedia Question-Answering Project}
% \descript{| CS 224N}
% \location{Mar 2020 | Language: Python}
% \begin{tightemize}
%     \item Enhanced Google’s ALBERT language model with a custom PyTorch “verifier” that answers fact questions from Wikipedia passage snippets.
%     \item Achieved 85\% F1 accuracy on \href{https://rajpurkar.github.io/SQuAD-explorer/}{SQuAD 2.0 challenge}.
% \end{tightemize}
% \sectionsep

\runsubsection{Optimized Task Scheduling Project}
\descript{| CS 221}
\location{Nov - Dec 2019 | Language: Python | \href{https://github.com/dastratakos/CS-221-Final-Project}{\faGithub $ $ GitHub repository$\link{}$}}
\begin{tightemize}
    \item Created a reinforcement learning model using value iteration and Q-learning to optimize revenue and customer satisfaction for service-based businesses.
\end{tightemize}
\sectionsep

\runsubsection{Pac-Man and Autonomous Car Assignments}
\descript{| CS 221}
\location{Nov 2019 | Language: Python}
\begin{tightemize}
    \item Implemented pathfinding algorithms to dictate optimal actions in various maps.
    \item Concepts included minimax, alpha-beta pruning, and Bayesian Networks.
\end{tightemize}
\sectionsep

% \runsubsection{Heap Allocator}
% \descript{| CS 107}
% \location{Mar 2019 | Language: C}
% \begin{tightemize}
%     \item Designed implicit and explicit allocators, balancing utilization and throughput.
% \end{tightemize}
% \sectionsep

%%%%%%%%%%%%%%%%%%%%%%%%%%%%%%%%%%%%%%
%     ACTIVITIES
%%%%%%%%%%%%%%%%%%%%%%%%%%%%%%%%%%%%%%

\section{Activities} 
\runsubsection{Stanford University Varsity Tennis Team}
\descript{| Member}
\location{Sep 2018 - present}
\begin{tightemize}
    \item Balance 20+ hours/week of training as a member of a Division I team ranked in the top 10 nationally with a full academic course load.
\end{tightemize}
\sectionsep

\runsubsection{Student-Athlete Advisory Cmte}
\descript{| Member, Social Events}
\location{Sep 2019 - present}
\begin{tightemize}
    \item Direct events with 100+ attendees to strengthen student-athlete community
    \item Maintain open communication between administration and student-athletes
\end{tightemize}
\sectionsep

\runsubsection{Abacus at Jin’s Mental Arithmetic Academy}
\descript{}
\location{2007 - 2018}
\begin{tightemize}
    \item Learned to use the abacus – a Japanese tool for fast mental math calculations. 
    \item 1st place in Mental Dictations at the international level | 2015, 2016, 2017.
\end{tightemize}

\end{minipage} 
\end{document}  \documentclass[]{article}
